\chapter{Исследовательская часть}

В данном разделе будут приведены результаты работы разработанного программного обеспечения и поставлен эксперимент по сравнению времени, затраченного на рендеринг одного кадра во время симуляции водной поверхности при различных размерах исходной сетки, разном количестве полигонов и накладываемых волн.

\section{Результаты работы программного обеспечения}

На рисунке \ref{ex1} приведён кадр симуляции волны по сетке размером 600х600, с амплитудой равной 40, скоростью 20, длиной волны 50, кривизной 0.02, вектором направления (1,1), количеством полигонов 10000, а также с двумя наложенными волнами, имеющими направление (1,0) и (0,1), а также продемонстрирован интерфейс программы.

\begin{figure}[H]
	\begin{center}
		\includegraphics[scale=0.28]{pictures/ex1.png}
	\end{center}
	\captionsetup{justification=centering}
	\caption{}
	\label{ex1}
\end{figure}

\newpage

На рисунке \ref{ex2} приведён кадр симуляции волны по сетке размером 300х300, с амплитудой равной 35, скоростью 25, длиной волны 100, кривизной 0.003, вектором направления (1,1), количеством полигонов 10000, наложение волн отсуствует.

\begin{figure}[H]
	\begin{center}
		\includegraphics[scale=1.0]{pictures/ex2.png}
	\end{center}
	\captionsetup{justification=centering}
	\caption{}
	\label{ex2}
\end{figure}


На рисунке \ref{ex3} приведён кадр симуляции волны по сетке размером 300х300, с амплитудой равной 40, скоростью 8, длиной волны 100, кривизной 0.8, вектором направления (1,1), количеством полигонов 10000, а также с двумя наложенными волнами, имеющими направление (1,0) и (-1,2).

\begin{figure}[H]
	\begin{center}
		\includegraphics[scale=0.8]{pictures/ex3.png}
	\end{center}
	\captionsetup{justification=centering}
	\caption{}
	\label{ex3}
\end{figure}


На рисунке \ref{ex4} приведён кадр симуляции волны по сетке размером 300х300, с амплитудой равной 35, скоростью 18, длиной волны 100, кривизной 0.28, вектором направления (1, 1), количеством полигонов 10000, а также с четыремя наложенными волнами, имеющими направление (1, 1.2), (1, 1.5), (1, 0) и (1, 0.8).

\begin{figure}[H]
	\begin{center}
		\includegraphics[scale=0.8]{pictures/ex4.png}
	\end{center}
	\captionsetup{justification=centering}
	\caption{}
	\label{ex4}
\end{figure}

На рисунке \ref{ex6} приведён кадр симуляции волны по сетке размером 600х600, с амплитудой равной 15, скоростью 20, длиной волны 50, кривизной 0.38, вектором направления (1, 1), количеством полигонов 4900, а также с девятью наложенными волнами, имеющими направление (1, 1.2), (1, 1.5), (0, 1), (-1, 2), (-1, -1), (1, 1), (1, 1), (1, 1), (-1, 1.2)б а также разные длины волн и скорости.

\begin{figure}[H]
	\begin{center}
		\includegraphics[scale=0.6]{pictures/ex6.png}
	\end{center}
	\captionsetup{justification=centering}
	\caption{}
	\label{ex6}
\end{figure}

\newpage

\section{Постановка эксперимента}

\subsection{Цель эксперимента}


Целью эксперимента является проведение трех независимых сравнений результатов работы программного обеспечения:

\begin{itemize}
	\item сравнение времени, затраченного на рендеринг одного кадра при разных размерах исходной сетки;
	\item сравнение времени, затраченного на рендеринг одного кадра при разном количесве полигонов;
	\item сравнение времени, затраченного на рендеринг одного кадра при разном количесве накладываемых волн;
\end{itemize}

\subsection{Cравнение времени, при разных размерах исходной сетки}

Сравнить время рендеринга при разом размере базовой сетки можно, если не изменять количество полигонов, на которое она разбивается, а также количество накладываемых волн. Взятое количество полигонов равно 10000 и количество накладываемых волн 3.

Результаты сравнения времени рендеринга кадра для изображения водной поверхности при разных размерах начальной сетки приведены в таблице \ref{tab:noise}.

\begin{table}[h!]
	\caption{}
	\label{tab:noise}
	\begin{center}
		\begin{tabular}{|c | c|} 
			\hline
			Размер сетки & Среднее время рендеринга кадра(в тиках) \\  
			\hline
			100x100 & 42782  \\
			\hline
			200x200 & 43284   \\
			\hline
			300x300 & 45594  \\
			\hline
			400x400 & 47417   \\
			\hline
			500x500 & 48903   \\
			\hline
			600x600 &  50219  \\
			\hline
			700x700 & 53101   \\
			\hline
			800x800 & 56732  \\
			\hline
		\end{tabular}
	\end{center}
\end{table}

\subsection{Cравнение времени, при разном количестве полигонов}

Сравнить время рендеринга при разом количестве полигонов  можно, если не изменять размер базовой сетки, а также количество накладываемых волн. Взятый размер стеки равен 400х400, количество накладываемых волн 3.

Результаты сравнения времени рендеринга кадра для изображения водной поверхности при разном количестве полигонов приведены в таблице \ref{tab:polygons}.

\begin{table}[h!]
	\caption{}
	\label{tab:polygons}
	\begin{center}
		\begin{tabular}{|c | c|} 
			\hline
			Количество полигонов & Среднее время рендеринга кадра(в тиках) \\  
			\hline
			100 & 14775  \\
			\hline
			400 &  15180 \\
			\hline
			900 &  17946 \\
			\hline
			1200 & 20605  \\
			\hline
			1600 &  21182  \\
			\hline
			2500 &  23760  \\
			\hline
			3600 &  25367 \\
			\hline
			4900 & 28089  \\
			\hline
			6400 &  34606 \\
			\hline
			8100 &  39220 \\
			\hline
			10000 & 47151 \\
			\hline
			12100 & 55233  \\
			\hline
			14400 & 65147  \\
			\hline
			16000 & 74444  \\
			\hline
			19600 & 86878  \\
			\hline
			22500 & 95676  \\
			\hline
		\end{tabular}
	\end{center}
\end{table}

\newpage

\subsection{Cравнение времени, при разном количестве накладываемых волн}


Сравнить время рендеринга при разом количестве накладываемых волн можно, если не изменять размер базовой сетки, а также количество полигонов. Взятый размер стеки равен 400х400, количество полигонов 10000.

Результаты сравнения времени рендеринга кадра для изображения водной поверхности при разном количестве накладываемых волн приведены в таблице \ref{tab:wave}.

\begin{table}[h!]
	\caption{}
	\label{tab:wave}
	\begin{center}
		\begin{tabular}{|c | c|} 
			\hline
			Количество накладываемых волн & Среднее время рендеринга кадра(в тиках) \\  
			\hline
			0 & 39753  \\
			\hline
			1 & 41244  \\
			\hline
			2 &  41453 \\
			\hline
			3 &  44749 \\
			\hline
			4 &  47229  \\
			\hline
			5 &  47883  \\
			\hline
			6 &  50893  \\
			\hline
			7 & 52283  \\
			\hline
			8 & 55608  \\
			\hline
			9 & 57292  \\
			\hline
			10 &  60908 \\
			\hline
			11 & 61400 \\
			\hline
			12 & 65104  \\
			\hline
		\end{tabular}
	\end{center}
\end{table}

\newpage
\section*{Вывод}

В данном разделе были рассмотрены примеры работы программного обеспечения и было сравнено время рендеринга при разном количестве полигонов, размере базовой сетки, количестве накладываемых волн. В результате сравнения были получены следующие результаты:

\begin{itemize}
	\item время рендеринга увеличивается с увелечением размера базовой секи
	\item время рендеринга увеличиваетсяс увелечение  количества полигонов
	\item время рендеринга увеличивается  с увелечением количества накладываемых волн
\end{itemize}

Как и ожидалось, с увелечением размера поверхности, количства полигонов и с усложнением волн(их накладывании) увеличивается время отрисовки.




