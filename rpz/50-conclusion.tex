\chapter*{Заключение}

Во время выполнения курсового проекта было реализованно программное обеспечение, которое позволяет симулировать поведение водной поверхности морей и океанов в реальном времени с использованием мощностей одного CPU.

Были проанализированы и рассмотренны существующие алгоритмы генерации водной поверхности. В качестве подходящего алгоритма, был выбран алгоритм волн Герстнера с импользованием сеточной модели построения водной поверхности. Для реалистичного построения изображения был выбран алгоритм удаления невидемых линий z-buffer, а также закраска методом Гуро. Данный выбор был сделан на основе анализа основных принципов построения реалистичного изображеия водной поверхности, которые позволили выбрать наиболее подходящее решение для поставленной задачи.

В процессе выполения данной работы были выполнены следующие задачи:

\begin{itemize}
	\item проанализированы существующие алгоритмы моделирования волн и выбраны подходящие для выполнения проекта
	\item проанализированы существующие алгоритмы освещения, позволяющие реализовать закраску с тенями, и выбраны подходящие для выполнения проекта
	\item реализован интерфейс программы
\end{itemize}

В процессе исследовательской работы было выяснено, что повышение детализации волн, а также размера плоскости симуляции повышает вычислительную сложность и время отрисовки одного кадра.

\addcontentsline{toc}{chapter}{Заключение}