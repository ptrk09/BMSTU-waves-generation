\chapter{Технологическая часть}

%\addcontentsline{toc}{section}{Технологическая часть}

\section{Выбор технических средств}

В качестве языка программирования, на котором будет реализовано программное обеспечение, выбран язык программирования C++ \cite{cpp}. Выбор языка обусловлен тем, что в нём есть поддержка объектно-ориентированной парадигмы, а также у для этого языка существует библиотека Qt \cite{qt}, наиболее полный функционал которой предостален именно для этого языка. 

Библиотека Qt была выбрана, потому что она имеет большое количество встроенных классов и большой функционал в работе с изображениями.

В качестве среды разработки выбран QtCreator \cite{qtc}, позволяющий легко работать с библиотекой  Qt и содержащий большое количество инструментов для языка программирования С++.

\section{Листинг кода}
Алгоритм разделён на модули: генерация волн Герстнера, обработчик волн, обработчик перемещения и поворотов, z-buffer с обработчиком освещения.

В листингах 3.1 – 3.3 приведен исходный код основных функций алгоритма симуляции водной поверхности.

\begin{lstinputlisting}[
	caption={Файл реализации модуля обработчик волн.},
	label={lst:mhc},
	style={cpp},
	]{../source/handler_waves.cpp}
\end{lstinputlisting}


\begin{lstinputlisting}[
	caption={Файл реализации генерации волн.},
	label={lst:mhc},
	style={cpp},
	]{../source/waves.cpp}
\end{lstinputlisting}


\begin{lstinputlisting}[
	caption={Файл реализации z-buffer с обработчиком освещения.},
	label={lst:mhc},
	style={cpp},
	]{../source/z_buffer.cpp}
\end{lstinputlisting}


\newpage


\subsection*{Вывод}
В данном разделе были рассмотренны средства реализации программного обеспечения и листинги исходных кодов программного обеспечения.