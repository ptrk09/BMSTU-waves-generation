\chapter{Технологическая часть}

%\addcontentsline{toc}{section}{Технологическая часть}

\section{Выбор технических средств}

В качестве языка программирования, на котором будет реализовано программное обеспечение, выбран язык программирования C++ \cite{cpp}. Выбор языка обусловлен тем, что в нём есть поддержка объектно-ориентированной парадигмы, а также для этого языка существует фреймворк Qt \cite{qt}, предоставляющая полный функционал, который нужен для создания пользовательского интерфейса программного обеспечения. 

В качестве среды разработки выбран QtCreator \cite{qtc}, позволяющий легко работать с фраймворком  Qt и содержащий большим количеством инструментов для различных языка программирования С++.

\section{Листинг кода}

В листингах 3.1 – 3.8 приведен исходный код реализации алгоритма симуляции водной поверхности.

Алгоритм разделён на модули: генерация волн Герстнера, обработчик волн, обработчик перемещения и поворотов, z-buffer с обработчиком освещения.


\begin{lstinputlisting}[
	caption={Заголовочный файл модуля обработчик волн.},
	label={lst:mhc},
	style={cpp},
	]{../handler_waves.h}
\end{lstinputlisting}


\begin{lstinputlisting}[
	caption={Файл реализации модуля обработчик волн.},
	label={lst:mhc},
	style={cpp},
	]{../handler_waves.cpp}
\end{lstinputlisting}


\begin{lstinputlisting}[
	caption={Заголовочный файл обработчика перемещения и поворотов.},
	label={lst:mhc},
	style={cpp},
	]{../handler_transform.h}
\end{lstinputlisting}


\begin{lstinputlisting}[
	caption={Файл реализации обработчика перемещения и поворотов.},
	label={lst:mhc},
	style={cpp},
	]{../handler_transform.cpp}
\end{lstinputlisting}


\begin{lstinputlisting}[
	caption={Заголовочный файл генерации волн.},
	label={lst:mhc},
	style={cpp},
	]{../waves.h}
\end{lstinputlisting}


\begin{lstinputlisting}[
	caption={Файл реализации генерации волн.},
	label={lst:mhc},
	style={cpp},
	]{../waves.cpp}
\end{lstinputlisting}


\begin{lstinputlisting}[
	caption={Заголовочный файл z-buffer с обработчиком освещения.},
	label={lst:mhc},
	style={cpp},
	]{../z_buffer.h}
\end{lstinputlisting}


\begin{lstinputlisting}[
	caption={Файл реализации z-buffer с обработчиком освещения.},
	label={lst:mhc},
	style={cpp},
	]{../z_buffer.cpp}
\end{lstinputlisting}


\newpage


\subsection*{Вывод}
В данном разделе были рассмотренны средства реализации программного обеспечения и листинги исходных кодов программного обеспечения.