	\chapter*{Введение}\addcontentsline{toc}{section}{Введение}
	
	Симуляция водной поверхности, которая способна преломлять и отражать свет, является одной из основных проблем при разработки игр или рендеринга видео. Одним из подходов построения изображения является реалистичное моделирование, нацеленное на максимальную схожесть с реальной морской поверхностью. Методы, позволяющие добиться данного эффекта, требуют огромное количество вычислительной мощности персональных компьютеров и используются в основном для кинопроизводства. Так же часто рассматривается вариант симуляции, при котором максимальная схожесть отходит на второй план, а в приоритете стоит производительность и вывод изображения в реальном времени. Данный способ применяется преимущественно в играх. В современное время специалисты стремятся добиться большей схожести изображения и анимации с реальным миром, при этом уменьшив сложность вычислений и время генерации.
	
	В рамках данной работы была поставлена задача реализовать симуляцию водной поверхности морей и океанов, которая будет работать в реальном времени и задействовать только вычислительные мощности cpu, на языке программирования с++. Решение поставленной задачи позволит людям внедрять модель океана в свои проекты, а также настраивать параметры симуляции по своему усмотрению.
	
	Несмотря на продолжительное существование данной проблемы, проектов, предназначены решить данную задачу на языке программирования с++, ещё не было. В сети Интернет в открытом доступе выложены решения, подходящие только для игровых движков, но все они не предназначены для проектов создаваемых с нуля.
	
	Для достижения поставленной цели в ходе работы требуется решить следующие задачи: 
	\begin{enumerate}
		\item проанализировать существующие алгоритмы моделирования волн и выбрать из них подходящие для выполнения проекта;
		\item проанализировать существующие алгоритмы освещения, позволяющие реализовать закраску с тенями, и выбрать из них подходящие для выполнения проекта;
		\item реализовать интерфейс программы.
	\end{enumerate}