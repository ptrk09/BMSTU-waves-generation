\chapter{Конструкторская часть}
%\addcontentsline{toc}{section}{Конструкторская часть}

\section{Подробный обзор волн Герстнера}

В своей статье \cite{gems1} компания Nvidia выделяет преимущество волн Герстнера, которое заключается в том, что данные волны, по сравнению  с обычными синусоидальными, дают более острые пики и более широкие впадины. Это достигается благодаря смещению не только но $z$ компоненту, но и по $x, y$. Так же для большей реалистичности используют наложение волн друг на друга. 

\section{Алгоритм получения высоты волны}

Функция (4) позволяет получить нам высоту одной конкретной синусоидальной волны. Рассмотрим её подробнее:

\begin{itemize}
	\item $A$ — амплитуда 
	\item $D$ — Двумерный вектор направления волны
	\item $x, y$ — координаты вершины
	\item $\omega$ -- частота волны
	\item $t$ — время
	\item $\varphi$ — фазовый угол
\end{itemize}

Выделим два новых параметра:

\begin{itemize}
	\item $L$ -- длина волны
	\item $V$ -- скорость волны
\end{itemize}

Выразим через них $\omega$ и $\varphi$:

\begin{equation*} 
\omega = \frac{2}{L}
\eqno(6)
\end{equation*}

\begin{equation*} 
\varphi = \frac{2V}{L}
\eqno(7)
\end{equation*}

Подставим формулы (6) и (7) в (4):
\begin{equation*} 
H(x, y, t) = \sum(A_{i} \times sin(D_{i} \cdot (x, y) \times \frac{2}{L_{i}} + t \times \frac{2V_{i}}{L_{i}}))
\eqno(8)
\end{equation*}

Остальные переменные, кроме времени и координат вершины, можно задавать, как произвольные константы.


\section{Смещение волн по трём компонентам.}

Как уже отмечалось ранее, волны Герстнера определяют смещения по компонентам $x, y, z$. Это помогает достичь реалистичного результата моделирования.

Рассмотрим волновую функцию Герстнера (5). Можно заметить, что компоненты получаемого вектора отвечают за смещения по каждой из координат. Так же компонента $z$ (высота гребня волны) данного вектора уже известна - это формула (8).

Подставим (8) в (5):

\begin{equation*} 
	P(x, y, t) = 
	\begin{pmatrix}
		x + \sum(Q_{i}A_{i} \times D_{i}x \times cos(\omega_{i}D_{i} \cdot (x, y) + \varphi_{i}t)),\\
		y + \sum(Q_{i}A_{i} \times D_{i}y \times cos(\omega_{i}D_{i} \cdot (x, y) + \varphi_{i}t)),\\
		\sum(A_{i} \times sin(D_{i} \cdot (x, y) \times \frac{2}{L_{i}} + t \times \frac{2V_{i}}{L_{i}}))
	\end{pmatrix}
	\eqno(9)
\end{equation*}

Получена конечная волновая функция Герстнера.


\section{Амплитуда и направление волн.}
Важным вопросом в моделировании волн являются определение амплитуды и направления.

Зависимость амплитуды волны от длины и текущих погодных условий, чаще всего задаются в сценарии во время разработки. Обычно вместе со средней длиной волны указывается медианная амплитуда, а для волны любого размера отношение её амплитуды к длине соответствует отношению средней амплитуды к средней длине волы.

Направление, по которому движется волна, полностью не зависит от других параметров. Следовательно выбор направления волны может происходить на основе любых критериев. Это всё также можно задать во время разработки.

\newpage
\section{Наложение волн.}

При моделировании большого водного пространства(океан или море) нам необходимы большие, острые пики. Для достижения большей реалистичности можно также использовать наложение волн. Чаще всего совмещают 4-12 волн (9-12 на океан или море, 4-8 на озеро). Т.к. нам необходима симуляция большой водной поверхности, следовательно необходимо выбирать 9-12 совмещений волн.

\begin{figure}[h]
	\center{\includegraphics[scale=0.6]{pictures/nalog_voln.jpg}}
	\caption{Примеры наложения волн.}
\end{figure}

\newpage
\section*{Вывод.}

Изучив научный материал по теме, было принято решение выражать частоту и фазовый угол волны через её скорость и длину. Так же количество волн для наложении было выбрано от 9 до 12 для моделирования больших водных гребней. Что касается амплитуды, направления, длины и скорости, то их значения будут задаваться константами внутри ПО. 
